\documentclass[output=paper,colorlinks,citecolor=green]{langscibook}

\author{Geisa Acácia Tavares\orcid{}\affiliation{Analista de Trânsito: Pedagoga} and Mário Diego Rocha Valente\orcid{}\affiliation{Analista de Trânsito: Estatístico}}
\title{Projeto Transitando no Bares} 
\abstract{Será apresentado um resumo teórico do projeto considerado o braço educativo da Lei seca no Estado od Pará, com resultados levandados em campo.}


\begin{document}
\maketitle


\section{Introdução}
O Projeto Transitando nos Bares teve início em junho de 2008, em virtude da entrada em vigor da Lei nº 11.705 (\textbf{Lei da Seca}).\vskip0.3cm

Uma equipe de Técnicos e Agentes de Educação da Coordenadoria de Educação do Detran/PA, realizou atividades com abordagens educativas, interagindo com o público que frequenta alguns bares do Município de Belém, capital do Estado do Pará, com o objetivo de sensibilizar essa população sobre o perigo existente na associação de álcool e direção. \vskip0.3cm

A avaliação dessas ações mostrou resultados positivos na medida em que as pessoas abordadas demonstraram ser favoráveis à ideia da campanha, além disso, os proprietários dos estabelecimentos visitados demonstraram o interesse de firmar parceria com o DETRAN/PA para trabalhar na prevenção da acidentalidade e mortalidade no trânsito, pois compreenderam a necessidade de campanhas que estimulem o cumprimento da lei associado à Educação para o Trânsito que é um fator indispensável na preservação da vida.\vskip0.3cm

Em 2009, as atividades continuaram sendo realizadas em alguns bares em Belém, sendo trabalhadas em quatro meses deste ano, com boa receptividade pelo público. O projeto foi interrompido por dois anos, entretanto, com a revisão da Lei 12.760 de dezembro de 2012 e a aplicação da resolução nº 432, de 23 de Janeiro de 2013, que tornaram mais rígidas as penalidades sobre o condutor flagrado sob o efeito de bebida alcoólica, as discussões sobre a reforma desse trabalho de conscientização foram retomadas.\vskip0.3cm

Em julho de 2015, no município de \textbf{Salinópolis} foi feito o Projeto Piloto, onde ocorreu uma intensa campanha de conscientização de veranistas, que com o apoio de donos de bares e restaurantes, aliado a outras estratégias contribuiu para um veraneio sem acidentes com vítimas fatais. A partir dos resultados iniciais de Salinópolis, o Projeto foi executado em Belém e nos Municípios de Bragança, Tucuruí, Marabá, Parauapebas, Santarém, Abaetetuba, Altamira e Itaituba.\vskip0.3cm


Em Belém o projeto foi realizado na Área 1: Av. V. Souza Franco (Entorno da “Doca”, onde foram atendidos os bares Meu Garoto, Du Pará e Doca Grill, Dom Bar, Devassa e Old School. No ano de 2017 o projeto será expandido para as demais áreas e para outros Municípios Pólos no Interior do Estado.


\section{Material e Métodos}
\subsection{Localização da Area de Estudo}

O Projeto Transitando nos Bares foi realizado no município de Belém e estruturado em ações que iniciaram com a definição da equipe executora, a partir do qual foram elencados os bares mais movimentados desta cidade, num total de três bares, que foram distribuídos em quatro áreas de atuação. \vskip0.3cm


Em seguida, planejou-se uma visita técnica aos estabelecimentos da Área 1 - Av. Visconde de Souza Franco e entorno, para “vender a ideia” do projeto aos proprietários de bares e fomentar o apoio dos mesmos na campanha, sendo então assinado um termo de aceite, mencionando a adesão à campanha de caráter social que não envolve transferências de recursos e que permite a divulgação das imagens nas mídias sociais e outros meios de comunicação, bem como a uso de peças publicitárias no interior do estabelecimento comercial.\vskip0.3cm 

Após a autorização, é feita a fixação de cartazes de banheiro, colocação de banners e adesivos com mensagens da campanha, com os devidos esclarecimentos sobre o projeto e informado todos os procedimentos que são feitos no momento da abordagem, inclusive com o uso do etilômetro, ocasião que fica agendado o dia e horário para a equipe fazer a incursão de sensibilização de orientação.\vskip0.3cm 

Além dos bares, foi estabelecida parceria com a maior associação de taxistas da primeira área de atuação, onde foi visitado os 2 pontos de táxis, sendo explicado sobre a importância social da campanha e o apoio necessário da categoria para divulgação do projeto, através da fixação de adesivos da campanha como “SE BEBER , CHAME UM TÁXI”.\vskip0.3cm

\subsection{Etapas Operacionais}

As abordagens são feitas em três etapas. Primeiramente, 10 agentes de educação foram divididos em três grupos, onde em cada grupo era colocado um integrante mais experiente, que fazia a apresentação da equipe e explicava sobre o Projeto Transitando nos Bares, interagindo de forma descontraída com as pessoas, deixando bem claro no momento da abordagem que se trata de uma campanha educativa, visando à conscientização sobre a utilização de bebida alcoólica associada à direção de veículo, que na atualidade é um dos principais fatores de riscos de acidentes de trânsito. \vskip0.3cm

Estimulava-se também nas abordagens o consumo consciente e responsável da bebida alcoólica, orientando para não dirigir nesse momento, podendo optar por eleger o \textbf{Motorista da Rodada}, em Belém conhecido como o \textbf{motora pai d’égua!}, bem como \textbf{Se for beber chame um táxi!}.\vskip0.3cm 

Na abordagem, a equipe foi orientada a evitar entrar em outros assuntos polêmicos ou gerar conflitos, uma vez que somente é dada continuidade na abordagem quando o cliente sinalizava positivamente que concorda em receber as informações.






\end{document}
