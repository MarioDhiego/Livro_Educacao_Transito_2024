\documentclass[output=paper,colorlinks,citecolor=brown]{langscibook}

\author{Geisa Acácia Tavares\orcid{}\affiliation{Analista de Trânsito: Estatístico} }

\title{Agente Multiplicador em Educação de Trânsito}
\abstract{}


\IfFileExists{../localcommands.tex}{
   \addbibresource{../localbibliography.bib}
   \input{../localpackages}
   \input{../localcommands}
   \input{../localhyphenation}
   \boolfalse{bookcompile}
   \togglepaper[23]%%chapternumber
}{}

\begin{document}
\maketitle

\section{Introdução} 

A violência no trânsito é uma das principais causas de morte tanto no contexto mundial quanto no nacional. De acordo com a Organização das Nações Unidas par Educação, Ciência e Cultura (Unesco) e a Secretaria Nacional de Trânsito (Senatran), valendo-se dos dados do Ministério da Saúde de 2021, cerca de 47.000 pessoas foram vítimas fatais de sinistros de trânsito no país, dentre essas, a maioria são de jovens com idade entre 16 e 29 anos, é importante ressaltar que 3.000 crianças morrem anualmente e outras 70.000 são hospitalizadas em consequência de acidentes de trânsito, sendo que 90\% destes acidentes poderiam ser evitados com ações preventivas.\vskip0.3cm

A educação para o trânsito mostrou-se um instrumento eficaz para a diminuição dos sinistros nas vias públicas, pois compreende-se que esta contribui direta e indiretamente para a redução do número daqueles, além de construir uma cultura de paz no espaço público, porém, para obter essa eficácia e tal redução, é preciso pensar na educação para o trânsito de uma forma holística que permita a reflexão do indivíduo e o faça compreender que o trânsito mais seguro deriva da consciência a nível global, pois uma nova cultural só será instalada a partir do ponto em que todos trabalhem para um objetivo comum. “É utópico pensar que só um ator social é responsável pelas atrocidades cometidas diariamente, ora se de maneira macroscópica não convém este olhar quiçá poderíamos analisar assim o trânsito” Faria e Braga (2004, p. 26).\vskip0.3cm

Entendendo a gravidade da violência devido à ausência de uma educação de trânsito, a equipe de servidores da Coordenadoria de Educação de Trânsito (CED) do Departamento de Trânsito do Estado do Pará (DETRAN-PA) baseando na Lei nº 9394/96 de diretrizes e bases da educação nacional art. 26 que cita: \vskip0.3cm

\begin{quote}
      Os currículos do ensino fundamental e médio devem ter uma base nacional comum, a ser complementada, em cada sistema de ensino e estabelecimento escolar, por uma parte diversificada, exigida pelas características regionais e locais da sociedade, da cultura, da economia e da clientela. (BRASIL, 1996).\vskip0.3cm
\end{quote}


Em conjunto com a Lei nº 9503/97 que institui o Código de Trânsito Brasileiro (CTB): 

\begin{quote}
    […] o Ministério da Educação e do Desporto, mediante proposta do CONTRAN e do Conselho de Reitores das Universidades Brasileira, diretamente ou mediante convênio, promoverá: 1. adoção em todos os níveis de ensino, de um currículo interdisciplinar com conteúdo programático sobre a segurança de trânsito. (BRASIL, 1997).
\end{quote}


\vskip0.3cm
Não obstante, valendo-se desses instrumentos legais e das experiências de educação para o trânsito, o Governo do Estado do Pará, através do Departamento de Trânsito do Estado do Pará-DETRAN-PA representado pela Coordenadoria de Educação de Trânsito percebeu a necessidade de adaptar a educação por meio da elaboração do “Curso Formação de Agente Multiplicador em Educação para o Trânsito”, visando socializar o conhecimento teórico/prático com o intuito de fomentar ações educativas no trânsito, dentro e fora do contexto educacional.\vskip0.3cm

\begin{quote}
    […] Não importa com que faixa etária trabalhe o educador ou a educadora. O nosso é um trabalho realizado com gente, miúda, jovem ou adulta, mas gente em permanente processo de busca. Gente formando-se, mudando, crescendo, reori\-en\-tan\-do-se, melhorando, mas porque gente, capaz de negar os valores, de dis\-tor\-cer-se, de recuar, de transgredir. Não sendo superior ou inferior a outra prática profissional, a minha, que é a prática docente, exige de mim um alto nível de responsabilidade ética de que a minha própria capacitação científica faz parte [...] (FREIRE, 1996 p. 53).\vskip0.3cm
\end{quote}


Sendo os jovens as principais vítimas dessa violência são notórias perceber que os atores sociais mais competentes para implementar a temática trânsito na cultura delas, são os gestores, coordenadores e docentes, já que podem implementar esta temática no contexto interdisciplinar do currículo escolar, nas práticas dos profissionais de trânsito e da saúde, logos esses atores tornam-se foco principal do projeto.\vskip0.3cm

Confrontando a realidade dos dados supracitados, esse projeto elencou o município de Abaetetuba para aplicar um plano piloto, pois em uma visita prévia ficaram evidentes diversos problemas, uma vez que não existia conselho municipal de trânsito, assim como ausência de convênios entre o município com a Polícia Militar do Estado do Pará e o DETRAN-PA, refletindo um quantitativo insuficiente de Agentes envolvidos no Trânsito. Facilitando a presença de veículos, sobretudo motocicletas sem o devido licenciamento ou com excesso de passageiros, condutores que não utilizam os dispositivos de segurança ou menores de idade conduzindo motocicletas. Ações intensificadas quando há grandes eventos no município.\vskip0.3cm

Como o plano piloto em Abaetetuba, contribui para a formação dos profissionais da educação, saúde e do trânsito, possibilitando conhecimentos de temas de educação para o trânsito, legislação, diretrizes de educação e de trânsito, gestão de trânsito, mobilidade urbana, políticas públicas e práticas voltadas para a Educação de Trânsito, contribuindo para a melhoria e qualidade de vida no Trânsito. Nesse sentido, a Coordenadoria de Educação de Trânsito (CED) está implementando a formação nos centos e quarenta e quatro municípios do Estado, até o momento já atendeu 98 municípios, com resultados positivos, conforme analisa a CED, que acredita que a Educação é a base de tudo e o começo é na escola.\vskip0.3cm

\begin{quote}
        A escola é um espaço de relações. Neste sentido, cada escola é única, fruto de sua história particular, de seu projeto e de seus agentes. Como lugar de pessoas e de relações, é também um lugar de representações sociais. Como instituição social ela tem contribuído tanto para a manutenção quanto para a transformação social. Numa visão transformadora ela tem um papel essencialmente crítico e criativo. A escola não é só um lugar para estudar, mas para se encontrar, conversar, confrontar-se com o outro, discutir, fazer política. Deve gerar insatisfação com o já dito, o já sabido, o já estabelecido (...) a escola não é só um espaço físico. É, acima de tudo, um modo de ser, de ver. (GADOTTI, 2007 p. 11 – 12).
\end{quote}



\newpage
\section{Material e Métodos}

As transformações mundiais refletem os comportamentos da sociedade e suas novas estruturas apontando caminhos para a sobrevivência. Como as pessoas vivem neste mundo globalizado, a educação passa a ser uma prioridade para a vida em sociedade. Educar, hoje, significa preparar as pessoas para viver em meio a toda a complexidade da globalização. \vskip0.3cm

O Código de Trânsito Brasileiro (CTB) vigente, nos seus 20 capítulos e os seus 341 artigos a palavra educação pode ser lida “vinte e oito vezes”, além de 13 palavras e termos correlatos (aprendizagem, campanhas educativas, especialização, nível de ensino, currículo de ensino, currículo interdisciplinar, escola pública, etc.) que aparecem vinte e uma vezes, o que representa 15\% dos 341 artigos da Lei. \vskip0.3cm

Educação para o Trânsito se consolida através do próprio CTB que teve o texto atualizado pela Lei Nº. 9.503, de 23 de setembro de 1997 e que no seu Capítulo VI – Da Educação para o Trânsito estabelece no Art. 74. “A educação para o trânsito é direito de todos e constitui dever prioritário para os componentes do Sistema Nacional de Trânsito” (BRASIL, 2002, p.31). \vskip0.3cm

No século XXI, mais precisamente em junho de 2009, que as Diretrizes Nacionais da Educação para o Trânsito na Pré-Escola e no Ensino Fundamental são aprovadas pelo DENATRAN, hoje denominado SENATRAN. Diante dessa construção, as Diretrizes Curriculares Nacionais para o Ensino de 9 (nove) anos, instituídas pelo parecer 011-2010 CNE/CEB, publicado no D.O.U. de 9/12/2010, Seção 1. São entendidas como um conjunto de orientações para as escolas, professores, educandos e suas famílias, bem como para os órgãos executivos e normativos das redes de sistema de ensino.\vskip0.3cm

As Diretrizes Curriculares do Ensino de 9 (nove) anos, contextualizam a temática do trânsito e referendam a sua promoção nos currículos, quando mencionam outras leis que complementam a Lei de Diretrizes e Bases da Educação (LDB) 9394/96- Base Nacional Comum Curricular (BNCC) de 12 de dezembro de 2017, como o Código de Trânsito Brasileiro (CTB), por exemplo, abrindo a porta para que as escolas possam promover a educação para o trânsito em seus currículos e nos projetos político pedagógicos, prevista nas leis que regem a educação.\vskip0.3cm

A BNCC referendada pelo Plano Nacional de Educação/PNE-2014 é o documento que define os aprendizados fundamentais durante toda a trajetória do aluno, desde a Educação Infantil até o Ensino Médio. Trata-se, portanto, de uma ferramenta que orienta e guia a elaboração e a atualização dos currículos escolares, funcionando como uma referência dos objetivos de aprendizagem em cada etapa da formação dos estudantes. Vale ressaltar que as particularidades sociais, regionais e metodológicas de cada instituição de ensino são consideradas neste documento, singularizando em cada currículo.\vskip0.3cm

Nesse sentido, as escolas poderão através da BNCC, incluir o Tema Trânsito com toda sua complexidade, no currículo, pois ela possibilita a discussão coletiva, é uma prerrogativa de cada escola, que deve, a partir daí desenvolver sua própria proposta curricular, sua identidade, respeitando a BNCC e a garantia do direito de aprendizagem. A educação para o trânsito na BNCC, faz parte dos Temas Contemporâneos Transversais (TCTs), no item Cidadania e Civismos. Reforçando a importância das escolas no sentido de desenvolver atividades que possibilitem análise e a reflexão sobre o trânsito, o qual está subentendido dentro das competências da BNCC.\vskip0.3cm

Esses documentos oferecem maior autonomia curricular às escolas e isso favorece para a elaboração de uma proposta pedagógica tanto à Escola como para a Coordenadoria de Educação de Trânsito, pois a CED objetiva ser referência e obter a excelência nos serviços oferecidos à sociedade, assim como elevar a qualidade de educação para o trânsito.  Para isso é necessário que a escola seja o ponto de partida, para desenvolver alguns programas e projetos voltados para o público escolar. Foi nesse contexto que a Coordenadoria de Educação (CED) elaborou e executa o Curso de Formação de Agente Multiplicador em Educação para o Trânsito.\vskip0.3cm

Pois percebeu-se que um dos principais tripés do trânsito, a educação, é uma das principais ferramentas para a diminuições dos riscos de sinistros no trânsito, a educação para o trânsito esclarece as leis e normas de trânsito e contribui para o comportamento adequado no trânsito, visando à preservação da vida e diminuição da violência no trânsito.\vskip0.3cm 

A educação estimula à cooperação, a solidariedade, a ética, a cidadania, sua base estão na busca do ser humano na superação das crises, pela consciência reflexiva e pela ação transformadora necessária à transposição dos limites. Brandão (1995, p. 10) afirma que “A educação é, como outras, uma fração do modo de vida dos grupos sociais que a criam e recriam, entre tantas outras invenções de sua cultura, em sua sociedade”. É no ensinar e aprender e no aprender para ensinar, que a educação, na maioria das vezes participa os processos de produção de crenças e valores refletindo na mobilidade do ser humano, no ato de andar ou conduzir um veículo passando de geração em geração. Isso porque a educação transforma as forças produtivas, criando na independência do agir novos valores, funcionando como mecanismos de mudança social, pois as pessoas ao se transformarem, também modificam o seu entorno.

\begin{quote}
    O papel essencial da Educação é o desenvolvimento contínuo do ser humano e das sociedades como uma via que conduz a processos sociais harmoniosos e democráticos em que se respeitem os Direitos Humanos e as diversidades. O pressuposto nos leva a acreditar na possibilidade de reversão dos quadros alarmantes de acidentes veiculados diariamente, através da educação para o trânsito em prol do desenvolvimento humano. (MELO, 2012, p. 3).
\end{quote}

Nesse sentido, a Coordenadoria de Educação de Trânsito do DETRAN-PA desenvolve o Curso de Formação de Agente Multiplicador em Educação para o Trânsito com objetivo de apresentar aos professores, coordenadores, gestores da rede de ensino, aos educadores não formais da comunidade; profissionais do trânsito, saúde e da segurança uma proposta de ação educativa continuada, através do Curso de Formação, onde possam implementar à Educação para o Trânsito como prática profissional, através de conteúdos e propostas pedagógicas que possibilitam uma reflexão crítica sobre as problemáticas do trânsito, para que a mobilidade urbana e humana flua com segurança e qualidade de vida para as pessoas na via, bem como os objetivos específicos: 


\begin{itemize}
    \item Permitir a adoção de atitudes seguras pelos professores e profissionais do trânsito, da saúde, da segurança e da comunidade;
    \item Possibilitar a reflexão sobre os fatores de risco no trânsito; 
    \item Formar professores e técnicos pedagógicos das escolas estaduais, municipais e particulares para facilitar aplicação do conceito trânsito através de metodologias no processo interdisciplinar, conforme estabelece a LDB/BNCC e CTB;
    \item Atender a demanda do Ministério Público do Pará e as solicitações de municípios;
    \item Atender, conforme o planejamento do Projeto, os municípios com maior índice de acidente e mortalidade no trânsito;
    \item Promover aos participantes do curso um conhecimento ampliado e crítico-reflexivo sobre o trânsito;
    \item Realizar parcerias com Secretária de Educação do Estado (SEDUC), Saúde (SESPA) e da SEAC/ Fundação Pará Paz - Programa de Governo Ter Paz, com Prefeituras através das Secretarias Municipais de Educação, Saúde, Segurança, Trânsito e outros órgãos do município;
    \item Estimular e motivar práticas educativas de trânsito, no ambiente do trabalho, da família e na comunidade dos envolvidos no curso;
    \item Proporcionar ao participante do curso elementos para a elaboração de planos de ação e projetos educacionais para a continuação e multiplicação de ações voltadas para o trânsito;
    \item Contribuir no aprendizado dos alunos com a inserção de uma cultura de segurança e prevenção de acidentes no trânsito;	
    \item Possibilitar a reflexão sobre os fatores de risco no trânsito, adotando medidas para minimizar as ocorrências de acidentes no município;
    \item Formar Agentes Multiplicadores que trabalham direta e indiretamente com Educação de Trânsito;
\end{itemize}

A metodologia do curso consistiu na construção do conhecimento, através de uma mediação dialógica, cujo objetivo foi informar, ouvir, discutir, refletir e construir sobre alguns conteúdos pertinentes, a fim de fornecer subsídios teóricos e práticos para o público-alvo, possibilitando que realizem ações educativas para o trânsito, criando uma rede de multiplicadores de educação para o trânsito e motivar educadores para trabalharem com o tema como proposta transversal a diferentes áreas da educação. \vskip0.3cm

Buscou-se como referencial teórico da metodologia, as práticas e ideologias da teoria de Paulo Freire, do aprender e o fazer, no processo que se dá na prática da realidade, para a realidade.\vskip0.3cm

Na visita prévia ao município, é selecionada uma amostra de 70 professores do universo de 25 escolas municipais e estaduais do ensino fundamental.  \vskip0.3cm

O curso é promovido de forma presencial, distribuídos em 5hora/aula ao dia com carga horária total de 40 horas, ao decorrer do curso os professores e os participantes recebem 02 kits de livros didáticos, jogos educativos, revistas e informativos, as instituições escolares também recebem 20 kits para contribuir no acervo da biblioteca a serem utilizados nas atividades e aplicados pelos docentes.\vskip0.3cm

Com o intuito de propor um curso voltado para conscientização, de maneira fomentar o uso dos livros didáticos distribuídos durante a formação dos agentes, pensou-se em disciplinas norteadas aos conceitos de Ética, Cidadania, Legislação, Comportamento Seguro, Educação para o Trânsito e Diretrizes e Metodologias aplicadas, todas vinculadas ao eixo trânsito, conforme disposto abaixo:















%\citet{Nordhoff2018} is useful for compiling bibliographies.


%\section*{Contributions}
%John Doe contributed to conceptualization, methodology, and validation.
%Jane Doe contributed to the writing of the original draft, review, and editing.

{\sloppy\printbibliography[heading=subbibliography,notkeyword=this]}
\end{document}
